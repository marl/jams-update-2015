% -----------------------------------------------
% Template for ISMIR Papers
% 2015 version, based on previous ISMIR templates
% -----------------------------------------------

\documentclass{article}
\usepackage{ismir,amsmath,cite}
\usepackage{graphicx}
\usepackage{xcolor}
\usepackage{listings}
\usepackage{url}
\usepackage{cleveref}
\usepackage{authblk}
\usepackage{booktabs}
\usepackage{bera}

\colorlet{punct}{red!60!black}
\definecolor{background}{HTML}{EEEEEE}
\definecolor{delim}{RGB}{20,105,176}
\colorlet{numb}{magenta!60!black}

\lstdefinelanguage{json}{
    basicstyle=\normalfont\ttfamily,
    numbers=left,
    numberstyle=\scriptsize,
    stepnumber=1,
    numbersep=8pt,
    showstringspaces=false,
    breaklines=true,
    frame=lines,
    backgroundcolor=\color{background},
    literate=
     *{0}{{{\color{numb}0}}}{1}
      {1}{{{\color{numb}1}}}{1}
      {2}{{{\color{numb}2}}}{1}
      {3}{{{\color{numb}3}}}{1}
      {4}{{{\color{numb}4}}}{1}
      {5}{{{\color{numb}5}}}{1}
      {6}{{{\color{numb}6}}}{1}
      {7}{{{\color{numb}7}}}{1}
      {8}{{{\color{numb}8}}}{1}
      {9}{{{\color{numb}9}}}{1}
      {:}{{{\color{punct}{:}}}}{1}
      {,}{{{\color{punct}{,}}}}{1}
      {\{}{{{\color{delim}{\{}}}}{1}
      {\}}{{{\color{delim}{\}}}}}{1}
      {[}{{{\color{delim}{[}}}}{1}
      {]}{{{\color{delim}{]}}}}{1},
}

\lstset{language=python,frame=none}
\lstset{language=json,frame=lines}

\renewcommand\Authfont{\bfseries}

% Title.
% ------
\title{JAMS v0.2 and beyond}

% Multiple authors
% To use with multiple author with possibly different addresses
% ---------------
%\multauthor
%{First author$^1$ \hspace{1cm} Second author$^1$ \hspace{1cm} Third author$^2$} { \bfseries{Fourth author$^3$ \hspace{1cm} Fifth author$^2$ \hspace{1cm} Sixth author$^1$}\\
  %$^1$ Department of Computer Science, University , Country\\
%$^2$ International Laboratories, City, Country\\
%$^3$  Company, Address\\
%{\tt\small CorrespondenceAuthor@ismir.edu, PossibleOtherAuthor@ismir.edu}
%}
%\def\authorname{First author, Second author, Third author, Fourth author, Fifth author, Sixth author}

% Single address
% To use with only one author or several with the same address
% ---------------
%\oneauthor
% {Names should be omitted for double-blind reviewing}
% {Affiliations should be omitted for double-blind reviewing}

% Two addresses
% --------------
%\twoauthors
%  {First author} {School \\ Department}
%  {Second author} {Company \\ Address}

% Three addresses
% --------------
%\threeauthors
%  {First author} {Affiliation1 \\ {\tt author1@ismir.edu}}
%  {Second author} {Affiliation2 \\ {\tt author2@ismir.edu}}
%  {Third author} {Affiliation3 \\ {\tt author3@ismir.edu}}

% Four addresses
% --------------
%\fourauthors
%  {First author} {Affiliation1 \\ {\tt author1@ismir.edu}}
%  {Second author}{Affiliation2 \\ {\tt author2@ismir.edu}}
%  {Third author} {Affiliation3 \\ {\tt author3@ismir.edu}}
%  {Fourth author} {Affiliation4 \\ {\tt author4@ismir.edu}}

\author[1,2,*]{Brian~McFee}
\author[2,3]{Eric~J. Humphrey}
\author[2,4]{Oriol~Nieto}
\author[2,5]{Justin~Salamon}
\author[2]{Rachel~Bittner}
\author[2]{Jon~Forsyth}
\author[2]{Juan~P. Bello}
\affil[1]{Center for Data Science, New York University}
\affil[2]{Music and Audio Research Laboratory, New York University}
\affil[3]{MuseAmi, Inc.}
\affil[4]{Pandora, Inc.}
\affil[5]{Center for Urban Science and Progress, New York University}

\def\authorname{Brian~McFee, Eric~J. Humphrey, Oriol~Nieto, Justin~Salamon,
    Rachel~M. Bittner, Jon~Forsyth, Juan~P. Bello}

\begin{document}
%
\maketitle
%
\let\oldthefootnote\thefootnote%
\renewcommand{\thefootnote}{\fnsymbol{footnote}}
\footnotetext[1]{Please direct correspondence to \url{brian.mcfee@nyu.edu}}
\let\thefootnote\oldthefootnote%
%
\begin{abstract}
This document describes the changes to the JSON Annotated Music Specification (JAMS)
format and implementation between v0.1 and v0.2.
\end{abstract}
%
\section{Introduction}\label{sec:introduction}

Throughout this document, the previous specification of JAMS as described by Humphrey
\emph{et al.}~\cite{jams2014} will be referred to as \emph{JAMS1}, while the current
specification will be referred to as \emph{JAMS2}.

\section{JAMS specification}\label{sec:schema}
In this section, we highlight the changes to the JAMS schema definition(s).
Since these changes apply to the file structure definition itself, they are independent of the software implementation used to parse or generate JAMS files.

\subsection{Tasks and namespaces}\label{sec:schema:annotations}

As illustrated in Figure~\cref{jams1}, JAMS1 took a \emph{task-major} approach to structuring annotations.  A collection of supported tasks was
defined within the JAMS1 schema, such as \emph{tag}, \emph{genre}, \emph{chord}, \emph{key}, \emph{melody}, \emph{etc}.  Annotations of a
particular task would then be accessed by indexing the array of annotations corresponding to that task, \emph{e.g.},
\begin{lstlisting}[language=python]
  >>> beat_annotation = jams_object.beat[0]
\end{lstlisting}

This structure is conceptually simple and easy to work with, but it poses several practical limitations.
First, it requires that all tasks be specified \emph{a priori} within the JAMS schema.
Consequently, each time a new task is introduced in the future, the core JAMS schema must be modified to accommodate it.
This is clearly undesirable, as it could lead to fragmentation of the JAMS specification if (when) different groups decide to extend the task definitions in one direction or another.

Second, it provides no means of distinguishing between different variations of a task.
As a simple example, take the case of \emph{tags}.
Different data sets are annotated using different vocabularies, which may be closed
(\emph{e.g.}, GTZAN or CAL500) or open (\emph{e.g.}, last.fm).
This implies that the validity of a tag annotation depends upon the target vocabulary, which is not explicitly coded within the schema.
(Indeed, an exhaustive coding of tag vocabularies within a fixed schema is impossible.) 
As a more nuanced example, \emph{chord} annotations can be drawn from different vocabularies (\emph{e.g.}, including or suppressing extensions),
or even radically different annotation styles, such as the pop-style annotations of Isophonics compared to the roman numeral annotations of the
Rock corpus.
In these cases, it is hardly sensible to group these variations together under a single task, since their annotations are not directly
comparable.




\subsection{Observations}\label{sec:schema:annotations}


\section{Implementation}\label{sec:implementation}

\subsection{Dynamic namespaces}\label{sec:imp:namespaces}
\subsection{Search}\label{sec:imp:search}
\subsection{Data frames}\label{sec:imp:dataframe}
\subsection{Validation}\label{sec:imp:validation}
\subsection{mir\_eval integration}\label{sec:imp:mireval}
\subsection{JAMZ compression}\label{sec:imp:compression}

\section{Future directions}\label{sec:future}

\subsection{Namespace conversions}
\subsection{Local namespaces and unstructured data}
\subsection{Partial annotations}

\bibliography{refs}

\end{document}
